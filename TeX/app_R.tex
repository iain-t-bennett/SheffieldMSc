\chapter{Selected R code}

\label{C:app_R}
This Appendix contains selected R \citep{Rsoftware} code needed to simulate the data used and to apply methods that are not available in standard software.

\section{Methods}
\label{S:app_R:sim1meth}
This section contains the code used to apply the methods described in Chapter \ref{CHAP:methods} that are not available in standard software.
\subsection{Sample data}
The following code creates an example dataset that is used to illustrate the methods here. The function \verb+sim.GenStudy+ can be found in Section \ref{S:app_R:simgen}.
\begin{Verbatim}[fontsize=\small, baselinestretch=0.75]
require(dplyr); require(survival)
exData  <- sim.GenStudy(seed = 1234, rho = 0.4, p.switch = 0.6)
# ITT Hazard Ratio and 95% CI
summary(coxph(Surv(os.t, os.e) ~ x.trt,data = exData))
\end{Verbatim}
\subsection{Rank Preserving Structural Failure Time}
An illustration of estimating a Hazard Ratio and 95\% CI using this method for the ``treatment group'' model with a log-rank test and estimating a hazard ratio using the comparison of $T_i$ and $U_i$ is shown below. Additional functions \verb+RPSFT.2pass+ and \verb+RPSFT.cox+ are defined below. 

\begin{Verbatim}[fontsize=\small, baselinestretch=0.75]
rpsft.input <- transmute(exData, 
                         event.time = os.t,
                         trt.ind    = x.trt, 
                         censor.ind = os.e, 
                         t.start    = ifelse(x.switch|x.trt, 
                                             pmax(0,t.switch,na.rm = TRUE), 
                                             0),
                         t.stop     = ifelse(x.switch|x.trt, 
                                                os.t, 
                                                0),
                         t.on       = t.stop - t.start, 
                         t.off      = os.t - (t.stop - t.start),
                         cutofftime = t.censor)
# do g-estimation
rpsft.output <- RPSFT.2pass(rpsft.input, Grho=0)
# use g-estimation results to calculate CI
summary(RPSFT.cox(rpsft.output, rpsft.input, Grho = 0)$cox.rpsft)
\end{Verbatim}
%dummy for syntax highlight$

\subsubsection{Function RPSFT.2pass}
The function \verb+RPSFT.2pass+ calls function \verb+RPSFT+ twice to perform two grid searches to identify the value of $\psi$ that minimises the chosen test statistic (specified by \verb+Grho+ which defines the test to use from the Harrington and Fleming family so \verb+Grho=0+ is log-rank and \verb+Grho=1+ is Wilcoxon).
\begin{Verbatim}[fontsize=\small, baselinestretch=0.75]
RPSFT.2pass <- 
  function(rpsft.input, 
           pass1.psimat = matrix(ncol = 1, data = 
                                 c(seq(from=-3, to=3, by=0.1))), 
           Grho = 0
){
  # apply RPSFT with wide grid
  pass1 <- RPSFT(rpsft.input, Grho=Grho, psimat = pass1.psimat)  
  
  # get interesting places to look (within 95 percent CI for z)
  pass1.limits <- pass1$psi.tried[abs(pass1$z) <= 1.96]
  pass2.psimat <- matrix(ncol = 1, data = 
          c(seq(from = min(pass1.limits), to = max(pass1.limits), by = 0.01)))  

  # apply RPSFT again
  pass2 <- RPSFT(rpsft.input, Grho=Grho, psimat = pass2.psimat)  

  # combine the grids searched for return
  pass.df <- rbind(data.frame(pass = 1, psi=pass1$psi.tried, z = pass1$z),
                   data.frame(pass = 2, psi=pass2$psi.tried, z = pass2$z))  
  pass.df <- summarise(group_by(pass.df, psi , z), pass = min(pass))  

  # return the second pass but add in the values tried in the first pass
  rc <- pass2
  rc$psi.tried <- pass.df$psi
  rc$z         <- pass.df$z
  rc$pass      <- pass.df$pass 
  return(rc)
}
\end{Verbatim}

\clearpage\newpage
\paragraph{Function RPSFT} performs g-estimation calling functions \verb+RPSFT.trypsi+ and \verb+RPSFT.latent+ repeatedly.

\begin{Verbatim}[fontsize=\small, baselinestretch=0.75]
RPSFT<-function(rpsft.input, psimat, Grho = 0){   
  # derive z values for a range of psi
  z <- apply(psimat,c(1),RPSFT.trypsi, Grho = Grho, rpsft.input = rpsft.input)  
  # chosen value for psi is that which has rank test Z value of 0
  # select based on when sign changes
  x <- sign(z)
  x1 <- x[-1]
  x2 <- x[-length(x)]  
  # selected psi index (either is 0 or is between a sign change)
  indx         <- which(z==0)
  indx_chg_low <- which(x1!= x2 & x1!=0 & x2!=0)
  indx_chg_hgh <- indx_chg_low+1
  psi.found    <- c(psimat[indx,1], 
                    0.5*(psimat[indx_chg_low,1]+psimat[indx_chg_hgh,1])) 
  # is it a unique solution?
  psi.unique <- as.numeric(length(psi.found)==1)  
  # apply proposal of White to take weighted average if is multiple
  psi.chosen <- sum(c(1,rep(c(-1,1),0.5*(length(psi.found)-1)))*psi.found)  
  latent <- RPSFT.latent(psi.chosen, rpsft.input)  
  # return the tested values, associated Z values, chosen psi and latent survival
  rc <- list(psi.tried  = as.numeric(psimat),
             z          = z,
             psi.found  = psi.found,
             psi.unique = psi.unique,
             psi.chosen = psi.chosen,
             latent     = latent
             )
  return(rc)
}
\end{Verbatim}

\paragraph{Function RPSFT.trypsi} estimates $Z(\psi)$ associated with a comparison of counterfactual survival times for a given value of $\psi$.
\begin{Verbatim}[fontsize=\small, baselinestretch=0.75]
RPSFT.trypsi <- function(psi, Grho, rpsft.input){ 
  latent <- RPSFT.latent(psi, rpsft.input)
  sv <- Surv(latent$event.time, latent$censor.ind)
  sd <- survdiff(sv ~ rpsft.input$trt.ind, rho = Grho)
  z  <- sign(sd$obs[2]-sd$exp[2])*sd$chisq^0.5
  return(z)
}
\end{Verbatim}
\paragraph{Function RPSFT.latent} derives latent survival times for a given value of $\psi$ while applying recensoring.
\begin{Verbatim}[fontsize=\small, baselinestretch=0.75]
RPSFT.latent <- function(psi, rpsft.input){
  latent<-data.frame(cstar = pmin(rpsft.input$cutofftime*exp(psi), 
                                  rpsft.input$cutofftime))
  latent$event.time <- pmin(rpsft.input$t.off + rpsft.input$t.on*exp(psi), 
                            latent$cstar )    
  latent$censor.ind <- as.numeric((latent$event.time < latent$cstar) 
                                   & rpsft.input$censor.ind)
  return(latent)
}
\end{Verbatim}
\clearpage\newpage
\subsubsection{Function RPSFT.cox}
The function \verb+RPSFT.cox+ estimates a Hazard Ratio and derives corrected test-based confidence intervals using the methods described in Section \ref{S:chap_methrev:RPSFTestHR}.
\begin{Verbatim}[fontsize=\small, baselinestretch=0.75]
RPSFT.cox <- function(rpsft.output, 
                      rpsft.input, 
                      Grho = 0, 
                      use.latent.only = TRUE
){ 
  # combine the observed and latent times
  df <- cbind(rpsft.input, 
              transmute(rpsft.output$latent, 
                        latent.event.time = event.time, 
                        latent.censor.ind = censor.ind, 
                        psi = rpsft.output$psi.chosen))  
  # derive counterfactual times from latent event times OR as T vs U
  if (use.latent.only){
    df <- mutate(df,
                 lat.on  = pmin(t.on * exp(psi), latent.event.time),
                 lat.off = ifelse((latent.event.time - lat.on) > 0, 
                                  latent.event.time - lat.on, 
                                  0),
                 cfact.time       = ifelse(trt.ind == 1, 
                                           lat.on * exp(-psi) + lat.off,  
                                           latent.event.time ), 
                 cfact.censor.ind = latent.censor.ind
    )
  } else {
    df <- mutate(df, 
                 cfact.time       = ifelse(trt.ind == 1, 
                                           event.time,  
                                           latent.event.time ),
                 cfact.censor.ind = ifelse(trt.ind == 1, 
                                           censor.ind,  
                                           latent.censor.ind )
    )                 
  }  
  # derive corrected hazard ratios
  cox.rpsft <- coxph(Surv(cfact.time, cfact.censor.ind) ~ trt.ind, data = df)  
  # correct standard error using Z value
  cox.rpsft$var.orig <- cox.rpsft$var
  z0 <- survdiff(Surv(event.time, censor.ind) ~ trt.ind, 
                 data = df, 
                 rho = Grho)$chisq ^ 0.5
  cox.rpsft$var <- matrix((abs(coef(cox.rpsft)) / z0)^2)  
  rc <- list(cox.rpsft = cox.rpsft, 
             cfact.time = df$cfact.time, 
             cfact.censor.ind = df$cfact.censor.ind, 
             psi.unique = rpsft.output$psi.unique)  
  return(rc)  
}
\end{Verbatim}
%dummy - correct highlight $
\clearpage
\newpage

\section{Code to Simulate data}
\label{S:app_R:simgen}
The function \verb+sim.GenStudy+ generates data as described in Chapter \ref{C:chap_sim_design}.

\begin{Verbatim}[fontsize=\small, baselinestretch=0.75]

sim.GenStudy <- function(
  # params modified by sim 
  simid         , # identifier
  seed          , # seed
  rho           , # correlation coefficient between switch time and os
  p.switch      , # proportion switch 
  prop.pd.int   , # proportion of patients with PFS before switch allowed
  beta_1a       , # treatment effect (as log(Hazard Ratio))
  beta_1b       , # treatment effect (as log(Hazard Ratio))
  beta_2a       , # treatment effect (as log(Hazard Ratio))
  beta_2b       , # treatment effect (as log(Hazard Ratio))
  # params not changed across all sims
  beta_pd       = log(0.4),  # treatment effect on TTP (as log(HR))  
  arm.n         = 250,       # patients per arm
  enroll.start  = 0,         # start of enrollment
  enroll.end    = 2,         # end of enrollment
  admin.censor  = 3,         # end of trial
  os.gamma      = 1.2,       # weibull shape - for OS
  os.lambda     = 0.3,       # weibull scale - for OS
  ttp.gamma     = 1.5,       # weibull shape - for TTP
  ttp.lambda    = 2          # weibull scale - for TTP
){
  set.seed(seed)  
  # combine so can select based on x 
  n_switch <- arm.n * p.switch  
  # u1 is just used to setup then not used
  # u2 and u3 are correlated uniform random variables used for OS and switch time
  # u4 used for censoring time (entry time)
  # u5 used for selecting switchers
  u <- matrix(nrow = arm.n*2, ncol = 5,data = runif(arm.n*2*5,0,1))
  # setup correlations using normal variables then convert back to uniform
  u[,3] <- pnorm(rho* qnorm(u[,2],0,1) + ((1-rho^2)^0.5)*qnorm(u[,1],0,1),0,1)  
  # generate covariates
  x.trt    <- rep(c(0,1), each = arm.n)  
  # generate os without treatment
  os.basis.t  <- (-(log(u[,2]) / (os.lambda)))^(1/os.gamma)  
  # generate ttp 
  ttp.untreated.t <- (-(log(u[,3]) / (ttp.lambda)))^(1/ttp.gamma)  
  # apply treatment effect
  ttp.treated.t <-(-(log(u[,3]) / (ttp.lambda*exp(beta_pd))))^(1/ttp.gamma)  
  # deive basis ttp
  ttp.basis.t <- ifelse(x.trt==1, ttp.treated.t, ttp.untreated.t)
  # generate pfs
  pfs.basis.t <- pmin(os.basis.t, ttp.basis.t)
  # generate entry time
  t.entry <- enroll.start + runif(u[,4])*(enroll.end-enroll.start)
  # generate follow up time
  t.censor <- admin.censor - t.entry
  # define a time in trial after which switch happens
  t.start.switch <- as.numeric(quantile(t.entry + pfs.basis.t, prop.pd.int))
  # who can switch - have ttp before os and censor
  can.switch <- as.numeric(
    ttp.basis.t < os.basis.t &
      ttp.basis.t < t.censor &
        t.entry+ttp.basis.t >= t.start.switch
  )[x.trt==0]  
  rank <- rep(arm.n,arm.n)
  # choose randomly from those who can switch to get the
  # required  number of switchers
  rank[can.switch==1] <- order(u[x.trt==0 & can.switch==1,5])
  x.switch <- c(as.numeric(rank<=n_switch), rep(0,arm.n))
  # calculate how many actually switch 
  actual.p.switch <- sum(x.switch) / arm.n
  # generate switch time
  t.switch <- rep(NA,arm.n*2)
  t.switch[x.switch==1] <- ttp.basis.t[x.switch==1]
  
  # get control non switch os
  os.orig.t <- rep(NA,arm.n*2)
  os.orig.t[x.trt == 0] <- (-(log(u[x.trt==0,2])) / (os.lambda ))^(1/os.gamma)
  
  # get experimental arm os
  t0  <- ttp.basis.t 
  t0v <- t0 ^ os.gamma
  c1 <- os.lambda * exp(beta_1a)
  c2 <- os.lambda * exp(beta_1b)
  v1 <- 1/os.gamma
  os.orig.t[x.trt == 1] <- ifelse(-log(u[x.trt==1,2])  < c1 * t0v[x.trt==1],
                                 (-log(u[x.trt==1,2]) / c1) ^ v1,
                                 ((-log(u[x.trt==1,2]) -
                                      c1*t0v[x.trt==1] + 
                                      c2*t0v[x.trt==1]) / c2) ^ v1
                                  )

  # get switch treatment (start from orig)
  os.cross.t <- os.orig.t
  t1 <- ttp.basis.t
  t2 <- ttp.basis.t + ttp.treated.t
  t1v <- t1^os.gamma
  t2v <- t2^os.gamma
  c3 <- os.lambda*exp(beta_2a)
  c4 <- os.lambda*exp(beta_2b)
  d1 <- x.switch==1 & -log(u[,2]) < os.lambda*t1v
  d3 <- x.switch==1 & -log(u[,2]) >= os.lambda*t1v + c3*(t2-t1)
  d2 <- x.switch==1 & -log(u[,2]) >= os.lambda*t1v & !d3
  os.cross.t[d1] <- (-log(u[d1,2]) / os.lambda )^v1
  os.cross.t[d2] <- ((-log(u[d2,2])- os.lambda*t1v[d2] + 
                      c3*t1v[d2]) / c3 )^v1
  os.cross.t[d3] <- ((-log(u[d3,2]) -
                      os.lambda*t1v[d3] - 
                      c3*(t2v[d3] - 
                      t1v[d3]) +c4*t2v[d3]) / c4 )^v1
  
  # apply censoring
  os.b.t <- pmin(os.basis.t, t.censor)
  os.b.c <- as.numeric(os.b.t==t.censor)
  os.o.t <- pmin(os.orig.t, t.censor)
  os.o.c <- as.numeric(os.o.t==t.censor)
  os.t <- pmin(os.cross.t, t.censor)
  os.c <- as.numeric(os.t==t.censor)
  # store ttp 
  ttp.t <- pmin(ttp.basis.t, t.censor)
  ttp.c <- as.numeric(ttp.t==t.censor)
  # derive PFS (min of os and ttp) 
  pfs.t <- pmin(os.t, ttp.t)
  pfs.c <- as.numeric(pfs.t==t.censor)
  # create a data frame
  patid <- 1:(arm.n*2)  
  df <- data.frame(patid    = patid,
                   x.trt    = x.trt,
                   x.switch = x.switch,
                   t.switch = t.switch,
                   t.censor = t.censor,
                   ttp.t    = ttp.t,
                   ttp.e    = 1 - ttp.c,
                   pfs.t    = pfs.t,
                   pfs.e    = 1 - pfs.c,
                   os.b.t   = os.b.t,
                   os.b.t   = 1 - os.b.c,
                   os.t     = os.t,
                   os.e     = 1 - os.c,
                   os.o.t   = os.o.t,
                   os.o.e   = 1 - os.o.c,
                   seed     = seed,
                   rho      = rho,
                   prop.pd.int = prop.pd.int,
                   p.switch = p.switch,
                   beta_1a = beta_1a,
                   beta_1b = beta_1b,
                   beta_2a = beta_2a,
                   beta_2b = beta_2b,
                   simid   = simid,
                   actual.p.switch = actual.p.switch
  )
  return(df)
}
\end{Verbatim}